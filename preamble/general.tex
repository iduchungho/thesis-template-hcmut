\RequirePackage[l2tabu,orthodox]{nag}
\RequirePackage{iftex}

\newcommand{\papersizeswitch}[3]{\ifnum\strcmp{\papersize}{#1}=0#2\else#3\fi}
%Change here the paper format
\papersizeswitch{b5paper}{\def\classfontsize{10pt}}{\def\classfontsize{12pt}}
\documentclass[\classfontsize,\papersize,twoside,showtrims,extrafontsizes]{memoir}
\usepackage{xhfill}
%%%%%%%%%%%%%%%%%%%%%%%%%%%%  PAPER SETTINGS    %%%%%%%%%%%%%%%%%%%%%%%%%%%%%%%%%
\showtrimsoff
\papersizeswitch{b5paper}{
    % Stock and paper layout
    \pagebv
    \setlrmarginsandblock{26mm}{20mm}{*}
    \setulmarginsandblock{35mm}{30mm}{*}
    \setheadfoot{8mm}{10mm}
    \setlength{\headsep}{7mm}
    \setlength{\marginparwidth}{18mm}
    \setlength{\marginparsep}{2mm}
}{
    \papersizeswitch{a4paper}{
        \pageaiv
        \setlength{\trimtop}{0pt}
        \setlength{\trimedge}{\stockwidth}
        \addtolength{\trimedge}{-\paperwidth}
        \settypeblocksize{634pt}{448.13pt}{*}
        \setulmargins{4cm}{*}{*}
        \setlrmargins{*}{*}{0.66}
        \setmarginnotes{17pt}{51pt}{\onelineskip}
        \setheadfoot{\onelineskip}{2\onelineskip}
        \setheaderspaces{*}{2\onelineskip}{*}
    }{}
}

\ifnum\strcmp{\showtrims}{true}=0
    % For printing B5 on A4 with trimmarks
    \showtrimson
    \papersizeswitch{b5paper}{\stockaiv}{\stockaiii}
    \setlength{\trimtop}{\stockheight}
    \addtolength{\trimtop}{-\paperheight}
    \setlength{\trimtop}{0.5\trimtop}
    \setlength{\trimedge}{\stockwidth}
    \addtolength{\trimedge}{-\paperwidth}
    \setlength{\trimedge}{0.5\trimedge}
    

    \trimLmarks
    
    % put jobname in left top trim mark
    \renewcommand*{\tmarktl}{%
      \begin{picture}(0,0)
        \unitlength 1mm
        \thinlines
        \put(-2,0){\line(-1,0){18}}
        \put(0,2){\line(0,1){18}}
        \put(3,15){\normalfont\ttfamily\fontsize{8bp}{10bp}\selectfont\jobname\ \
          \today\ \ 
          \printtime\ \ 
          Page \thepage}
      \end{picture}}

    % Remove middle trim marks for cleaner layout
    \renewcommand*{\tmarktm}{}
    \renewcommand*{\tmarkml}{}
    \renewcommand*{\tmarkmr}{}
    \renewcommand*{\tmarkbm}{}
\fi

\checkandfixthelayout                 % Check if errors in paper format!
\sideparmargin{outer}                 % Put sidemargins in outer position (why the fuck is this option not default by the class?)
%%%%%%%%%%%%%%%%%%%%%%%%%%%%%%%%%%%%%%%%%%%%%%%%%%%%%%%%%%%%%%

% Shows boundaries overlaid on the actual pages. Comment it to don't see them! 
%\usepackage{showframe}


% Large environments
\usepackage{microtype}
\usepackage{mathtools}
\usepackage{listings}                 % Source code printer for LaTeX
\usepackage{tikz}

% Mathematics
\usepackage{amsmath,dsfont,amsfonts,braket}
\usepackage{physics}
\usepackage[shortlabels]{enumitem}

% Citation & Acronyms
\usepackage{cite}
\usepackage{acronym}
\usepackage[hang,flushmargin]{footmisc} % Remove footnote indentation

% Modify Parts, Chapters, Sections...
\usepackage{titlesec}
\usepackage{fancyhdr}

% Links
\usepackage[hyphens]{url}             % Allow hyphens in URL's
\usepackage[unicode=false,psdextra]{hyperref}                 % References package

% Graphics, PDF files and colors
\usepackage{pdfpages}
\usepackage{graphicx}                 % Including graphics and using colours
\usepackage{subcaption}     % To use the environment "subfigure"
\usepackage{xcolor,colortbl}        % Defined more color names
\usepackage{eso-pic}                  % Watermark and other bag
\usepackage{preamble/dtucolors}
\graphicspath{{graphics/}}


% Language & Date
\ifPDFTeX
  \usepackage[vietnamese,danish,english]{babel}
  \usepackage[utf8]{inputenc}
\else
  % \usepackage{polyglossia}    % multilingual typesetting and appropriate hyphenation
  % \setdefaultlanguage{vietnamese}
  % \setotherlanguage{english}
  \usepackage[utf8]{inputenc}
  \usepackage[vietnamese]{babel}
\fi

\let\ordinal\relax % Avoids a warning due to incompatibility between Memoir class and <datetime>
\usepackage[nodayofweek]{datetime}
\usepackage[super]{nth}
\newdateformat{mydate}{\nth{\THEDAY}{ }\monthname[\THEMONTH] \THEYEAR} % Personal date style

% Floating objets, captions and references
\usepackage{flafter}  % floats is positioned after or where it is defined! 
%\setfloatlocations{figure}{bhtp}   % Set floats for all figures
%\setfloatlocations{table}{bhtp}    % Set floats for all tables
%\setFloatBlockFor{section}         % Typeset floats before each section
\usepackage[noabbrev,nameinlink,capitalise]{cleveref} % Clever references. Options: "fig. !1!" --> "!Figure 1!"
\DeclareCaptionFont{dtu}{\normalsize\sffamily\selectfont} %\fontsize{10.5}{10.5}
\usepackage[labelfont={dtu,bf},labelsep=period]{caption}
\captionsetup{font=small}
\captionnamefont{\bfseries}
\subcaptionlabelfont{\bfseries}
\newsubfloat{figure}
\newsubfloat{table}
%\letcountercounter{figure}{table}         % Consecutive table and figure numbering
%\letcountercounter{lstlisting}{table}     % Consecutive table and listings numbering
% strip things from equation references, making them "(1)" instead of "Equation~1"
% from http://tex.stackexchange.com/questions/122174/how-to-strip-eq-from-cleveref
\crefformat{equation}{(#2#1#3)}
\crefrangeformat{equation}{(#3#1#4) to~(#5#2#6)}
\crefmultiformat{equation}{(#2#1#3)}%
{ and~(#2#1#3)}{, (#2#1#3)}{ and~(#2#1#3)}

% Table of contents (TOC)
\setcounter{tocdepth}{3}              % Depth of table of content
\setcounter{secnumdepth}{-1}           % Depth of section numbering
\setcounter{maxsecnumdepth}{3}        % Max depth of section numbering


% Partstyle
\titleformat{\part}[display]{\filcenter\chapnamefont\fontsize{42pt}{0pt}\selectfont\setlength{\parskip}{-1.5cm}}{\vspace{4cm}\color{dtugray}{\chapnamefont\fontsize{36pt}{0pt}\selectfont}
\partname\chapnamefont\color{dtured}\fontsize{40pt}{0pt}\selectfont\hspace{.3em}\thepart}{4ex}{}
\titlespacing{\part}{0pt}{0pt}{0pt}


% Chapterstyle
\makeatletter
\makechapterstyle{mychapterstyle}{
    \chapterstyle{default}
    \def\format{\normalfont\sffamily}
    \setlength\beforechapskip{0mm}
    \renewcommand*{\chapnamefont}{\format\HUGE}
    \renewcommand*{\chapnumfont}{\format\fontsize{54}{54}\selectfont}
    \renewcommand*{\chaptitlefont}{\format\fontsize{42}{42}\selectfont}
    \renewcommand*{\printchaptername}{\chapnamefont\MakeUppercase{\@chapapp}}
    \patchcommand{\printchaptername}{\begingroup\color{dtugray}}{\endgroup}
    \renewcommand*{\chapternamenum}{\space\space}
    \patchcommand{\printchapternum}{\begingroup\color{dtured}}{\endgroup}
    \renewcommand*{\printchapternonum}{%
        \vphantom{\printchaptername\chapternamenum\chapnumfont 1}
        \afterchapternum}
    \setlength\midchapskip{1ex}
    \renewcommand*{\printchaptertitle}[1]{\raggedleft \chaptitlefont ##1}
    \renewcommand*{\afterchaptertitle}{\vskip0.5\onelineskip \hrule \vskip1.3\onelineskip}
}
\makeatother
\chapterstyle{mychapterstyle}


% Appendix page style
\renewcommand*{\cftappendixname}{Appendix\space} % Adds the word "Appendix" in ToC

\let\appendixpagenameorig\appendixpagename
\renewcommand{\appendixpagename}{\vspace{-2cm}\normalfont\sffamily\format\fontsize{42pt}{0pt}\selectfont\appendixpagenameorig} % Change font for the "Appendices" page

\renewcommand\appendixtocname{\large Appendices} % Change font dimension and name for "Appendices" in ToC

%%%%%%%%%%%%%%%%%%%%%%% Defines the protocol environment %%%%%%%%%%%%%%%%%%%%%

\newcounter{protocol}
\newenvironment{protocol}[1]
   {\par\addvspace{\topsep}
   \noindent
   \tabularx{\linewidth}{@{}>{\columncolor{dtured!8}[0pt][\tabcolsep]} X @{}}
   %{@{}>{\columncolor{gray!15}[0pt][\tabcolsep]}l|>{\columncolor{blue!25}}l|X|X@{}}
   %{@{} X @{}}
    %\hline
    \rule{0pt}{2.5ex}
    \refstepcounter{protocol}\textbf{\cellcolor{dtured!25}\large\sffamily Protocol \theprotocol} \large\sffamily#1 \\
    %\hline
    }
 { \\
    %\hline
   \endtabularx
   \par\addvspace{\topsep}}
%%%%%%%%%%%%%%%%%%%%%%%%%%%%%%%%%%%%%%%%%%%%%%%%%%%%%%%%%%%%%%%%%%%%%%%%

% Header and footer
\def\hffont{\sffamily\small}
\makepagestyle{myruled}
\makeheadrule{myruled}{\textwidth}{\normalrulethickness}
\makeevenhead{myruled}{\hffont\thepage}{}{\hffont\leftmark}
\makeoddhead{myruled}{\hffont\rightmark}{}{\hffont\thepage}
\makeevenfoot{myruled}{}{}{}
\makeoddfoot{myruled}{}{}{}
%%%%%% Adds the word "Appendix" to appendices header %%%
\makepagestyle{appruled}
\makeheadrule{appruled}{\textwidth}{\normalrulethickness}
\makeevenhead{appruled}{\hffont\thepage}{}{\hffont\appendixname~\leftmark}
\makeoddhead{appruled}{\hffont\appendixname~\rightmark}{}{\hffont\thepage}
\makeevenfoot{appruled}{}{}{}
\makeoddfoot{appruled}{}{}{}
%%%%%%%%%%%%%%%%%%%%%%%%%%%%%%%%%%%%%%%%%%%%%%%%%%%%%%%%
\makepsmarks{myruled}{
    \nouppercaseheads
    \createmark{chapter}{both}{shownumber}{}{\space}
    \createmark{section}{right}{shownumber}{}{\space}
    \createplainmark{toc}{both}{\contentsname}
    \createplainmark{lof}{both}{\listfigurename}
    \createplainmark{lot}{both}{\listtablename}
    \createplainmark{bib}{both}{\bibname}
    \createplainmark{index}{both}{\indexname}
    \createplainmark{glossary}{both}{\glossaryname}
}
\pagestyle{myruled}
\copypagestyle{cleared}{myruled}      % When \cleardoublepage, use myruled instead of empty
\makeevenhead{cleared}{\hffont\thepage}{}{} % Remove leftmark on cleared pages
\makeevenfoot{plain}{}{}{}            % No page number on plain even pages (chapter begin)
\makeoddfoot{plain}{}{}{}             % No page number on plain odd pages (chapter begin)


% \*section, \*paragraph font styles
\setsecheadstyle              {\huge\sffamily\raggedright}
\setsubsecheadstyle           {\LARGE\sffamily\raggedright}
\setsubsubsecheadstyle        {\Large\sffamily\raggedright}
%\setparaheadstyle             {\normalsize\sffamily\itseries\raggedright}
%\setsubparaheadstyle          {\normalsize\sffamily\raggedright}


% Hypersetup
\hypersetup{
    %pdfauthor={\thesisauthor{}},
    %pdftitle={\thesistitle{}},
    pdfdisplaydoctitle,
    bookmarksnumbered=true,
    bookmarksopen,
    breaklinks,
    linktoc=all,
    plainpages=false,
    unicode=true,
    colorlinks=false,
    citebordercolor=dtured,           % color of links to bibliography
    filebordercolor=dtured,           % color of file links
    linkbordercolor=dtured,           % color of internal links (change box color with linkbordercolor)
    urlbordercolor=blue,               % color of external links
    hidelinks,                        % Do not show boxes or colored links.
}



% Bibliography appearance in Table of Contents
\makeatletter
\renewcommand{\@memb@bchap}{%
  \ifnobibintoc\else
    \phantomsection
    \addcontentsline{toc}{part}{\bibname}%
  \fi
  \chapter*{\bibname}%
  \bibmark
  \prebibhook
}
\let\oldtableofcontents\tableofcontents
\newcommand{\newtableofcontents}{
    \@ifstar{\oldtableofcontents*}{
        \phantomsection\addcontentsline{toc}{chapter}{\contentsname}\oldtableofcontents*}}
\let\tableofcontents\newtableofcontents
\makeatother


% Confidential
\newcommand{\confidentialbox}[1]{
    \put(0,0){\parbox[b][\paperheight]{\paperwidth}{
        \begin{vplace}
            \centering
            \scalebox{1.3}{
                \begin{tikzpicture}
                    \node[very thick,draw=red!#1,color=red!#1,
                          rounded corners=2pt,inner sep=8pt,rotate=-20]
                          {\sffamily \HUGE \MakeUppercase{Confidential}};
                \end{tikzpicture}
            }
        \end{vplace}
    }}
}


% Prefrontmatter
\newcommand{\prefrontmatter}{
    \pagenumbering{alph}
    \ifnum\strcmp{\confidential}{true}=0
        \AddToShipoutPictureBG{\confidentialbox{10}}   % 10% classified box in background on each page
        \AddToShipoutPictureFG*{\confidentialbox{100}} % 100% classified box in foreground on first page
    \fi
}


% % DTU frieze
% \newcommand{\frieze}{%
%     \AddToShipoutPicture*{
%         \put(0,0){
%             \parbox[b][\paperheight]{\paperwidth}{%
%                 \includegraphics[trim=130mm 0 0 0,width=0.9\textwidth]{DTU-frise-SH-15}
%                 \vspace*{2.5cm}
%             }
%         }
%     }
% }

% % This is a double sided book. If there is a last empty page lets use it for some fun e.g. the frieze.
% % NB: For a fully functional hack the \clearpage used in \include does some odd thinks with the sequence numbering. Thefore use \input instead of \include at the end of the book. If bibliography is used at last everything should be ok.
% \makeatletter
% % Adjust so gatherings is allowd for single sheets too! (hacking functions in memoir.dtx)
% \patchcmd{\leavespergathering}{\ifnum\@memcnta<\tw@}{\ifnum\@memcnta<\@ne}{
%     \leavespergathering{1}
%     % Insert the frieze
%     \patchcmd{\@memensuresigpages}{\repeat}{\repeat\frieze}{}{}
% }{}
% \makeatother


%% Sharelatex fix for microtype warnings
\makeatletter
\def\MT@is@composite#1#2\relax{%
  \ifx\\#2\\\else
    \expandafter\def\expandafter\MT@char\expandafter{\csname\expandafter
                    \string\csname\MT@encoding\endcsname
                    \MT@detokenize@n{#1}-\MT@detokenize@n{#2}\endcsname}%
    % 3 lines added:
    \ifx\UnicodeEncodingName\@undefined\else
      \expandafter\expandafter\expandafter\MT@is@uni@comp\MT@char\iffontchar\else\fi\relax
    \fi
    \expandafter\expandafter\expandafter\MT@is@letter\MT@char\relax\relax
    \ifnum\MT@char@ < \z@
      \ifMT@xunicode
        \edef\MT@char{\MT@exp@two@c\MT@strip@prefix\meaning\MT@char>\relax}%
          \expandafter\MT@exp@two@c\expandafter\MT@is@charx\expandafter
            \MT@char\MT@charxstring\relax\relax\relax\relax\relax
      \fi
    \fi
  \fi
}
% new:
\def\MT@is@uni@comp#1\iffontchar#2\else#3\fi\relax{%
  \ifx\\#2\\\else\edef\MT@char{\iffontchar#2\fi}\fi
}
\makeatother



